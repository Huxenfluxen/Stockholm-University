\documentclass{article}
\usepackage[swedish]{babel}	
\usepackage[utf8]{inputenc} 		
\usepackage{amsmath} 

\begin{document}

% Kommentarer sätts efter procenttecken. De syns inte i den färdiga filen.


\title{Introduktion till \LaTeX}
\author{Qimh Xantcha}
\maketitle 

\noindent 
Att skriva avancerad matematik i \LaTeX~är både lätt och roligt.  
% Tilde-tecknet efter kommandot \LaTeX skapar ett litet mellanrum efteråt; 
% annars vill LaTeX dra ihop orden.






\section{Text}


En av finesserna med \LaTeX~är, att programmet automatiskt bryter raderna och till och med är kapabelt att avstava.


% Blank rad här ovanför ger nytt stycke.
\emph{Kursiverad text} skrivs så här. 
\textbf{Fetstilt text} skrivs så här. 

\bigskip
Här har vi skapat ett mellanrum i texten.






\section{Matematik}

\subsection{Korta och långa formler}

Korta formler, som till exempel Einsteins berömda formel från relativitetsteorien,
$E=mc^2$, typsätts bäst inuti texten.
Då används enkla dollartecken. 

Mer komplicerade formler typsätts bäst på en egen rad med hjälp av dubbla dollartecken:
$$
a^2 + b^2 = c^2.
$$



\subsection{Vanliga matematiska symboler}

Här följer nu en lista med exempel på hur vanliga matematiska symboler kan typsättas. 

\begin{enumerate}

\item Grekiska bokstäver: 
$$
\alpha, \beta, \gamma, \pi.
$$

\item Exponenter: 
$$
e^{\pi i} = -1.
$$

\item Bråk: 
$$
\frac{x}{y} + \frac{y}{x} = \frac{x^2+y^2}{xy}.
$$

\item Rötter: 
$$
\sqrt{x+y} \neq \sqrt{x} + \sqrt{y}, \qquad \sqrt[3]{x^3}=x.
$$

\item Olikheter: 
$$
\sqrt{ab} \leq \frac{a+b}{2} \qquad \text{då $a,b\geq 0$.} 
$$

\item Moduliräkning: 
$$
a^{p-1} \equiv 1 \bmod p.
$$

\item Elementära funktioner: 
$$
\cos^2 x + \sin^2 x = 1, \qquad \ln e^x = e^{\ln x} = x.
$$

\item Gränsvärden: 
$$
\lim_{x\to 0} \frac{\sin x}{x} = 1.
$$

\item Integraler: 
$$
\int_{0}^1 x^2 \ dx = \left[ \frac{x^3}{3} \right]_0^1 = \frac{1}{3}.
$$

% Symbolen \ ger ett litet mellanrum mellan funktionen x^2 och dx.
% Kommandona \left och \right nyttjas för att ge klamrarna rätt storlek.

\item Matriser: 
$$
\begin{pmatrix}
1 & 2 \\
3 & 4
\end{pmatrix}^\top
=
\begin{pmatrix}
1 & 3 \\
2 & 4
\end{pmatrix}.
$$

\item Vektorer: 
$$
\overrightarrow{AB} + \overrightarrow{BC} = \overrightarrow{AC}.
$$

\end{enumerate}

\subsection{Långa uträkningar}

Långa uträkningar måste kanske radbrytas manuellt: 
\begin{multline*}
\frac{d}{dx}\tan x = \frac{d}{dx}\frac{\sin x}{\cos x} 
= \frac{\cos x\cdot \cos x - \sin x \cdot (-\sin x)}{\cos^2 x} \\ % Här bryts raden.
= \frac{\cos^2 x + \sin^2 x}{\cos^2 x} 
= \frac{\cos^2 x}{\cos^2 x} + \frac{\sin^2 x}{\cos^2 x} 
= \tan^2 x + 1.
\end{multline*}




\end{document}